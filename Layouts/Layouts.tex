\title{Android -- Eine Einführung}
\subtitle{Layouts, Views \& Adapter}
\author[A. Wilhelm]{Andreas Wilhelm}
\institute[IfI -- Uni Göttingen]{
   Institut für Informatik\\ Georg-August-Universität Göttingen
}
\titlegraphic{}
%\date{\today}
\date{www.avedo.net}

\begin{frame}[plain]
  \titlepage
\end{frame}

\section[Contents]{}
\begin{frame}
	\frametitle{Contents}
	\tableofcontents[onlyparts]
\end{frame}

\part{Layouts}
\frame{\partpage}
\begin{frame}
	\frametitle{Contents}
	\tableofcontents[]
\end{frame}

\section{Überblick}
\begin{frame}
   \frametitle{Allgemeines}
   \begin{itemize}
      \item Oberflächen für Aktivitäten, Menüs, Dialogen und Widgets
      \item Grundelemente Views und ViewGroups (Layouts \& AdapterViews)
      \item Deklaration im XML-Format
      \item Formatierung eines Elements über Attribute
      \item Alternativ View-Objekte während der Laufzeit im Quellcode 
         erzeugen und ändern
      \item Weitestgehend einheitliche Namensgebung für XML-Attribute 
         und Objekt-Methoden
   \end{itemize}

   \begin{alertblock}{XML-Layouts}
      Deklaration von Layouts in XML verschlankt den Code, 
      ermöglicht Unterstützung verschiedener Oberflächen, Sprachen 
      und Auflösungen und erleichtert den Debugging-Prozess.
   \end{alertblock}
\end{frame}

\begin{frame}
   \frametitle{Deklaration}
   \begin{itemize}
      \item Erstellung von Layouts erinnert an HTML-Code
      \item Jedes layout darf nur ein Wurzelelement enthalten 
         (View, ViewGroup, Layout, Merge)
      \item Darunter beliebig verschachtelte Layout-Hierarchie
      \item Speichern von Layouts unter \emph{/res/layouts}
      \item Zugriff im Quellcode über automatisch generierte Klasse \emph{R}
      \item Zugriff in anderen Layouts über \emph{@[package:]layout/filename}
   \end{itemize}
\end{frame}

\section{ViewGroups}
\begin{frame}
   \frametitle{Allgemeines}
   \begin{itemize}
      \item Struktur-Elemente, die andere Views anordnen
      \item LinearLayout, das FrameLayout oder das RelativeLayout
      \item Weitere ViewGroups ohne Kindelemente (Bsp: AdapterView)
      \item Basis-Attribute ID (optional), sowie Breite und Höhe
   \end{itemize}

   \begin{alertblock}{Android-IDs}
      Eindeutige ID eines Layout-Elements wird über \emph{android:id} 
      in der Form \emph{@+id/name} deklariert. + signalisiert dabei, dass es sich 
      um eine neue ID handelt. Diese ID kann im Quellcode mit \emph{R.id.name} 
      referenziert werden.\\

      Alternativ kann eine ID als Resource deklariert und mit \emph{@id/name} 
      referenziert werden. Die Deklaration einer solchen Resource wird in XML vorgenommen 
      und unter einem beliebigen Dateinamen unter \emph{/res/values} abgelegt.\\

      \lstinputlisting[
         backgroundcolor=\color{boxcol},language=xml,
         caption=Deklaration von IDs,label={lst:id_resource.xml}]{src/xml/id_resource.xml}
   \end{alertblock}
\end{frame}

\begin{frame}
   \frametitle{Bemaßungen}
   \begin{itemize}
      \item Bemaßungen eines ViewGroups mit den Attributen \emph{android:layout\_width} 
			und \emph{android:layout\_height} entweder explizit oder implizit
      \item Explizite Angabe erfolgt dabei als Wert oder als Resource
      \item Implizite über die Schlüsselwörter \emph{fill\_parent}, 
         \emph{match\_parent} oder \emph{wrap\_content}
   \end{itemize}

   Zur Bemaßungen können verschiedene Einheiten (px, dp, sp, pt, in, mm) 
   oder Schlüsselwörter verwenden.\\

	\begin{attrDesc}{+p{3cm}|^p{7cm}}
		Schlüsselwort & Beschreibung\\
		\hline
      \emph{match\_parent} & Weist aktuellem Objekt die Größe des Eltern-Elements zu 
         (löst \emph{fill\_parent} ab)\\
      \emph{fill\_parent} & Weist aktuellen Objekt die Größe des Eltern-Elements zu\\
      \emph{wrap\_content} & Sorgt für Anpassung der Größe, sodass Inhalt umschlossen wird\\
  	\end{attrDesc}

   Alternativ kann man auch eine Resource im Quellcode über die Klasse \emph{R} 
   oder in einer XMl-Datei mit dem Schlüssel \emph{@[package:]dimen/dimension\_name} 
   referenziert.\\

   \lstinputlisting[
      backgroundcolor=\color{boxcol},language=xml,
      caption=Deklaration von Dimensionen,label={lst:dimen.xml}]{src/xml/dimen.xml}
\end{frame}

\section{View}
\begin{frame}
   \frametitle{Allgemeines}
   \begin{itemize}
      \item Allein stehendes Element einer grafischen Oberfläche
      \item Nimmt rechteckige Fläche des Bildschirms ein
      \item Kümmert sich um die Ausgabe und das Verarbeiten von Ereignissen
      \item Textfelder, Buttons und Eingabefelder, wie beispielsweise \emph{TextView}
      \item Deklaration per XML oder im Quellcode
      \item Verwaltung der Basis-Attribute, wie bei ViewGroups
   \end{itemize}
\end{frame}

\section{RequestFocus}
\begin{frame}
   \frametitle{RequestFocus}
   \begin{itemize}
      \item Gibt View bzw. ViewGroup direkt bei Initialisierung den Fokus
      \item XML-Deklaration als leeres Element \emph{\textless{}requestFocus\textgreater}
      \item Zuweisung eines solchen Elements nur einmal pro Datei
   \end{itemize}
\end{frame}

\section{Include}
\begin{frame}
   \frametitle{Include}
   \begin{itemize}
      \item Verbinden bereits existierender Layouts
      \item Element \emph{\textless{}include\textgreater}
      \item Einziges eigenes Attribut Name des einzubinden Layouts
      \item Optional eine ID, sowie Breite und Höhe des Layouts
      \item Werte überschreiben die Werte des Wurzelknotens im eingebundenen Layout
      \item Änderungen an den Bemaßungen mit \emph{android:layout\_width} und 
      	\emph{android:layout\_height} nur für beide Attribute möglich
   \end{itemize}

   \begin{alertblock}{ViewStub}
      Alternativ kann anstatt des \emph{\textless{}include\textgreater}-Elements 
      auch ein \emph{ViewStub} verwendet werden.
   \end{alertblock}
\end{frame}

\section{Merge}
\begin{frame}
   \frametitle{Merge}
   \begin{itemize}
      \item Wurzelelement um Layouts zu verbinden
      \item Sinnvoll wenn Layouts mit \emph{\textless{}include\textgreater} 
         verbunden werden sollen
      \item Flacherer und dadurch einfacherer Aufbau der View-Hierarchie
   \end{itemize}
\end{frame}
