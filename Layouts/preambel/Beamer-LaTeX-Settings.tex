% *** Sprache *****************************
\usepackage[utf8]{inputenc}
\usepackage[T1]{fontenc}
\usepackage[ngerman]{babel}

\usepackage[%
   %final,
   %draft % do not include images (faster)
]{graphicx}
\usepackage{subfigure}
 
\usepackage{xcolor}
\definecolor{boxheadcol}{RGB}{59, 91, 134}
\definecolor{boxcol}{gray}{.9}

\definecolor{avedo}{RGB}{59, 91, 134}
\definecolor{darkgray}{RGB}{69,69,69}
\definecolor{lightgray}{gray}{.9}
\definecolor{darkblue}{RGB}{59, 91, 134}
\definecolor{fgcgray}{rgb}{0.4, 0.4, 0.4}
\definecolor{bgctitle}{RGB}{59, 91, 134}
\definecolor{fgctitle}{rgb}{0.99, 0.99, 0.95}
 
\usepackage{tabularx}   % Erweiterte Tabellen Optionen
\usepackage{booktabs}
\usepackage{multicol}

\usepackage{amssymb,amsmath}
\usepackage{listings}
\usepackage{hyperref}
 
\definecolor{dkgreen}{rgb}{0,0.6,0}
\definecolor{gray}{rgb}{0.5,0.5,0.5}
\definecolor{mauve}{rgb}{0.58,0,0.82}

\lstset{ %
  language=Java, % the language of the code
  basicstyle=\tiny, % the size of the fonts that are used for the code
  numbers=left, % where to put the line-numbers
  numberstyle=\tiny\color{gray}, % the style that is used for the line-numbers
  stepnumber=5, % the step between two line-numbers. If it's 1, each line
  numbersep=5pt, % how far the line-numbers are from the code
  firstnumber=1,
  numberfirstline=false,
  backgroundcolor=\color{white}, % choose the background color. You must add \usepackage{color}
  showspaces=false, % show spaces adding particular underscores
  showstringspaces=false, % underline spaces within strings
  showtabs=false, % show tabs within strings adding particular underscores
  tabsize=3, % sets default tabsize to 2 spaces
  captionpos=b, % sets the caption-position to bottom
  breaklines=true, % sets automatic line breaking
  breakatwhitespace=false, % sets if automatic breaks should only happen at whitespace
  title=\lstname, % show the filename of files included with \lstinputlisting;
  keywordstyle=\color{blue}, % keyword style
  commentstyle=\color{dkgreen}, % comment style
  stringstyle=\color{mauve}, % string literal style
  escapeinside={\%*}{*)}, % if you want to add a comment within your code
  morekeywords={*,...}, % if you want to add more keywords to the set
  xleftmargin=2em,
  xrightmargin=2em,
  aboveskip=1em,
  nolol=true
}

\def\thelstlisting{}
\makeatletter
\AtBeginDocument{%
  \renewcommand \thelstlisting
       {}%
}
\makeatother
