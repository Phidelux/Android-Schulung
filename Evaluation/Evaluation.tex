\PassOptionsToPackage{table}{xcolor}
\documentclass{scrartcl}

% Include all needed base libraries.
\usepackage[ngerman]{babel} % we speak (new) german...:-)
%\usepackage[latin1]{inputenc} % input encoding ISO 8859-1 (Latin-1)
%\usepackage[latin9]{inputenc} % input encoding ISO 8859-15 (Latin-9) (includes euro symbol)
\usepackage[utf8]{inputenc}
\usepackage[T1]{fontenc} % font encoding T1
\usepackage{amssymb,amsmath}

% Include more packages.
\usepackage[left=3cm,right=3cm,top=2cm,bottom=2cm,includeheadfoot]{geometry}
\usepackage{coordsys,logsys,color}
\usepackage{multirow}
\usepackage{xcolor}
\usepackage{colortbl}

% Define spacing between lines.
\usepackage[onehalfspacing]{setspace}

% Define some meta data.
\title{Evaluation -- Android-Schulung}
\author{Andreas Wilhelm}

% Define colors for the title box.
\definecolor{darkblue}{RGB}{59, 91, 134}
\definecolor{fgcgray}{rgb}{0.4, 0.4, 0.4}
\definecolor{bgctitle}{RGB}{59, 91, 134}
\definecolor{fgctitle}{rgb}{0.99, 0.99, 0.95}

% Declare some short font tags.
\renewcommand{\familydefault}{cmss}
\newcommand{\tfont}[1]{\textcolor{fgctitle}{\fontfamily{cmss}\fontseries{bx}\fontshape{n}\fontsize{20.48}{0pt} \selectfont #1}}
\newcommand{\itfont}[1]{\textcolor{bgctitle}{\fontfamily{cmss}\fontseries{bx}\fontshape{n}\fontsize{13}{0pt} \selectfont #1}}

% Redeclare the make title.
\makeatletter
\def\@maketitle{
  \begin{center}
    \colorbox{bgctitle}{
      \parbox{\textwidth}{
        \vskip 8pt
        \centering{\tfont{\@title}}
        \par
        \vskip 8pt
      }
    }
    \colorbox{white}{
      \parbox{\textwidth}{
        \vskip 8pt
        \centering{\itfont{\@author}}
        \par
        \vskip 8pt
      }
    }
  \end{center}
  \vskip 2.5em
}
\makeatother

% Disable ident after paragraph.
\parindent 0pt

% Include packages used for the questionnaire.
\usepackage{wasysym}
\usepackage{enumitem}
\usepackage{color}
\usepackage{forloop}
\usepackage{ifthen}

% Declare some new commands based on this packages.
\newcommand{\Qq}[1]{\textbf{#1}}

% \QO = Circle or box to be ticked.
\newcommand{\QO}{$\Box$}

% \Qrating = Automatically create a rating scale with NUM steps.
\newcounter{qr}
\newcommand{\Qrating}[1]{\QO\forloop{qr}{1}{\value{qr} < #1}{---\QO}}

% \Qline = Again, this is very simple. It helps setting the line
% thickness globally. Used both by direct call and by \Qlines.
\newcommand{\Qline}[1]{\noindent\rule{#1}{0.6pt}}

% \Qlines = Insert NUM lines with width=\linewith.
\newcounter{ql}
\newcommand{\Qlines}[1]{\forloop{ql}{0}{\value{ql}<#1}{\vskip0em\Qline{\linewidth}}}

% \Qlist = This is an environment very similar to itemize.
\newenvironment{Qlist}{%
  \renewcommand{\labelitemi}{\QO}
  \begin{itemize}[leftmargin=1.5em,topsep=-.5em]
}{%
  \end{itemize}
}

%% \Qtab = A "tabulator simulation".
\newlength{\qt}
\newcommand{\Qtab}[2]{
  \setlength{\qt}{\linewidth}
  \addtolength{\qt}{-#1}
  \hfill\parbox[t]{\qt}{\raggedright #2}
}

%% \Qitem = Item with automatic numbering.
\newcounter{itemnummer}
\newcommand{\Qitem}[2][]{
  \ifthenelse{\equal{#1}{}}{\stepcounter{itemnummer}}{}
  \ifthenelse{\equal{#1}{a}}{\stepcounter{itemnummer}}{}
  \begin{enumerate}[topsep=2pt,leftmargin=2.8em]
  \item[\textbf{\arabic{itemnummer}#1.}] #2
  \end{enumerate}
}

%% \QItem = Like \Qitem but with alternating background color.
\definecolor{bgodd}{rgb}{0.8,0.8,0.8}
\definecolor{bgeven}{rgb}{0.9,0.9,0.9}
\newcounter{itemoddeven}
\newlength{\gb}
\newcommand{\QItem}[2][]{
  \setlength{\gb}{\linewidth}
  \addtolength{\gb}{-5.25pt}
  \ifthenelse{\equal{\value{itemoddeven}}{0}}{%
    \noindent\colorbox{bgeven}{\hskip-3pt\begin{minipage}{\gb}\Qitem[#1]{#2}\end{minipage}}%
    \stepcounter{itemoddeven}%
  }{%
    \noindent\colorbox{bgodd}{\hskip-3pt\begin{minipage}{\gb}\Qitem[#1]{#2}\end{minipage}}%
    \setcounter{itemoddeven}{0}%
  }
}

% Begin the document.
\begin{document}

% Include the headline.
\maketitle

Um die Qualität und den Praxisbezug der Schulung überprüfen zu können, möchte 
ich Sie bitten, sich einen kurzen Moment Zeit zu nehmen, um die folgenden 
Fragen zu beantworten. Vielen Dank für Ihre aktive Mithilfe zur Verbesserung 
dieser Schulung.\\

\section{Teilnehmer}

\Qq{Android-Erfahrung} \Qtab{6cm}{Profi \Qrating{5} Einsteiger}

\vspace{5mm}

\Qq{Java-Erfahrung} \Qtab{6cm}{Profi \Qrating{5} Einsteiger}

\vskip 3em

\section{Seminar}

\Qq{Ihre Erwartungen} \Qtab{6cm}{positiv \Qrating{5} negativ}

\vspace{5mm}

\Qq{Gesamt-Eindruck} \Qtab{6cm}{hervorragend \Qrating{5} katastrophal}

\vspace{5mm}

\Qq{Inhalte} \Qtab{6cm}{ansprechend \Qrating{5} uninteressant}

\vspace{5mm}

\Qq{Gliederung} \Qtab{6cm}{sehr gut \Qrating{5} keine}

\vspace{5mm}

\Qq{Lehrunterlagen} \Qtab{6cm}{hilfreich \Qrating{5} nutzlos}

\vspace{5mm}

\Qq{Umfang} \Qtab{6cm}{zu viele Inhalte \Qrating{5} zu wenig Inhalte}

\vspace{5mm}

\Qq{Praxis \& Theorie} \Qtab{6cm}{zu praktisch \Qrating{5} zu theoretisch}

\vspace{5mm}

\Qq{Pausen} \Qtab{6cm}{zu lang \Qrating{5} zu kurz}

\pagebreak

\section{Trainer}

\Qq{Fachwissen \& Kompetenz} \Qtab{6cm}{hohe Kompetenz \Qrating{5} keine Ahnung}

\vspace{5mm}

\Qq{Präsentation} \Qtab{6cm}{sehr gut \Qrating{5} mangelhaft}

\vspace{5mm}

\Qq{Sprache} \Qtab{6cm}{sehr gut \Qrating{5} mangelhaft}

\vspace{5mm}

\Qq{Teilnehmerorientierung} \Qtab{6cm}{sehr stark \Qrating{5} keine}

\vskip 3em

\section{Fazit}

\textbf{Würden Sie dieses Seminar weiterempfehlen?}\\
\hspace*{1cm}\QO{} Ja\hskip0.5cm \QO{} Nein

\vspace{5mm}

\textbf{Sind Fragen offen geblieben?}\\
\hspace*{1cm}\QO{} Ja\hskip0.5cm \QO{} Nein

\vskip 3em

\section{Anregungen}
\Qlines{5}
	
\end{document}
