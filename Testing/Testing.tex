\title{Android -- Eine Einführung}
\subtitle{Testen \& Veröffentlichen}
\author[A. Wilhelm]{Andreas Wilhelm}
\institute[IfI -- Uni Göttingen]{
   Institut für Informatik\\ Georg-August-Universität Göttingen
}
\titlegraphic{}
%\date{\today}
\date{www.avedo.net}

\begin{frame}[plain]
  \titlepage
\end{frame}

\part{Testen von Applikationen}
\frame{\partpage}
\begin{frame}
	\frametitle{Contents}
	\tableofcontents[]
\end{frame}

\section{Überblick}
\begin{frame}
	\frametitle{Allgemeines}
	\begin{itemize}
		\item Android-SDK bringt Emulator und Debugging-Monitor \emph{LogCat} 
			mit
		\item Integration von JUnit unter Android
		\item \emph{monkeyrunner}, eine auf Python basierende Stress-Test-API
		\item JUnit-Tests sind Emulator basiert
		\item Mock-Framework zwar theoretisch vorhanden praktisch unbrauchbar
			$\rightarrow$ Alternativ-Projekte interessant
	\end{itemize}
\end{frame}

\section{Aufbau von Unit-Tests}
\begin{frame}
	\frametitle{Allgemeines}
	\begin{itemize}
		\item Eigenes Android-Projekt mit Test-Paketen und -Klassen
		\item Eine Klasse implementiert die Tests eines Moduls
		\item Eine Klassen-Methode einen Test
		\item Android lädt Test- und das Applikations-Paket mit Test-Runner
	\end{itemize}
\end{frame}

\begin{frame}
	\frametitle{Test-Projekt Erstellen}
	\begin{itemize}
		\item Einbindung der SDK-Test-Tools über Eclipse ADT-Plugin
		\item Erstellen des Projekt-, Source- und Resource-Verzeichnisses
		\item Anlegen eines Test-Pakets mit Suffix \emph{.test}
		\item Generieren des Android-Manifests und der ANT-Dateien
		\item Eintragung des \emph{InstrumentationTestRunner} im Manifest
		\item Platzieren der Test-Klassen im Source-Ordner (\emph{src/})
	\end{itemize}
\end{frame}

\begin{frame}
	\frametitle{JUnit Basis}
	\begin{itemize}
		\item Java-Tests ohne Android-API-Zugriffe mit \emph{TestCase}
		\item Andernfalls Verwendung von \emph{AndroidTestCase}
		\item Spezialisierte Klassen, wie \emph{ApplicationTestCase}, 
			\emph{LoaderTestCase}, \emph{ProviderTestCase2} oder \emph{ServiceTestCase}
		\item Alle Klassen ermöglichen Verwendung von JUnit-Assertions
	\end{itemize}
\end{frame}

\section{Der Android Lifecycle}
