\title{Android -- Eine Einführung}
\subtitle{Testen \& Veröffentlichen}
\author[A. Wilhelm]{Andreas Wilhelm}
\institute[www.avedo.net]{}
\titlegraphic{}
\date{\today}
%\date{CSC Computer-Schulung \& Consulting GmbH}

\begin{frame}[plain]
  \titlepage
\end{frame}

\part{Testen von Applikationen}
\frame{\partpage}
\begin{frame}
	\frametitle{Contents}
	\tableofcontents[]
\end{frame}

\section{Überblick}
\begin{frame}
	\frametitle{Allgemeines}
	\begin{itemize}
		\item Android-SDK bringt Emulator und Debugging-Monitor \emph{LogCat} 
			mit
		\item Integration von JUnit unter Android
		\item \emph{monkeyrunner}, eine auf Python basierende Stress-Test-API
		\item JUnit-Tests sind Emulator basiert
		\item Mock-Framework zwar theoretisch vorhanden praktisch unbrauchbar
			$\rightarrow$ Alternativ-Projekte interessant
	\end{itemize}
\end{frame}

\section{Aufbau von Unit-Tests}
\begin{frame}
	\frametitle{Allgemeines}
	\begin{itemize}
		\item Eigenes Android-Projekt mit Test-Paketen und -Klassen
		\item Eine Klasse implementiert die Tests eines Moduls
		\item Eine Klassen-Methode einen Test
		\item Android lädt Test- und das Applikations-Paket mit TestRunner
	\end{itemize}
\end{frame}

\begin{frame}
	\frametitle{Test-Projekt Erstellen}
	\begin{itemize}
		\item Einbindung der SDK-Test-Tools über Eclipse ADT-Plugin
		\item Erstellen des Projekt-, Quellcode- und Ressourcen-Verzeichnisses
		\item Anlegen eines Test-Pakets mit Suffix \emph{.test}
		\item Generieren des Android-Manifests und der ANT-Dateien
		\item Eintragung des \emph{InstrumentationTestRunner} im Manifest
		\item Platzieren der Test-Klassen im Source-Ordner (\emph{src/})
	\end{itemize}
\end{frame}

\begin{frame}
	\frametitle{JUnit Basis}
	\begin{itemize}
		\item Java-Tests ohne Android-API-Zugriffe mit \emph{TestCase}
		\item Andernfalls Verwendung von \emph{AndroidTestCase}
		\item Spezialisierte Klassen, wie \emph{ApplicationTestCase}, 
			\emph{LoaderTestCase}, \emph{ProviderTestCase2} oder \emph{ServiceTestCase}
		\item Alle Klassen ermöglichen Verwendung von JUnit-Assertions
	\end{itemize}
\end{frame}

\section{Der Android Lifecycle}

\begin{frame}
	\frametitle{Instrumentation}
	\begin{itemize}
		\item Einflussnahme auf Lebenszyklus einer Applikation wichtig
		\item Android Klasse \emph{Instrumentation}
		\item Simulieren das beispielsweise Starten (\emph{onCreate()}) 
			oder Beenden (\emph{onDestroy}) einer Activity
		\item Beispielanwendung: Prüfung des gespeicherten Zustands einer Activity
		\item Spezialisierte Klassen, wie \emph{ActivityInstrumentationTestCase2}
	\end{itemize}
	
	\begin{alertblock}{AndroidTestCase}
		\emph{AndroidTestCase} bringt weitaus mehr als Methoden \emph{setUp()} und \emph{tearDown()} mit. 
		Sie enthält Methoden, die das Testen von Zugriffsrechten ermöglichen und 
		eine Methode, die Speicher-Lecks verhindert, indem sie Klassen-Referenzen entfernt.
	\end{alertblock}
\end{frame}

\begin{frame}
	\frametitle{Beispiel}
	\lstinputlisting[
		caption=Kontrolle des Lifecycle,
		label={lst:instrumentation.java}]{src/java/instrumentation.java}
\end{frame}

\begin{frame}
	\frametitle{Anmerkungen}
	\begin{itemize}
		\item Test-Paket und Activity werden nicht im selben Thread geladen
		\item Zugriff auf grafische Oberfläche mit \emph{runOnUiThread()}
		\item Alternative: Annotation \emph{@UiThreadTest}
	\end{itemize}
\end{frame}

\begin{frame}
	\frametitle{Beispiel}
	\lstinputlisting[
		caption=Die \emph{@UiThreadTest} Annotation,
		label={lst:uiThreadAnnotation.java}]{src/java/uiThreadAnnotation.java}
\end{frame}

\section{Basis-Komponenten und Erweiterungen}

\begin{frame}
	\frametitle{Anmerkungen}
	\begin{itemize}
		\item Bibliotheken für das Testen von Activities, Services und ContentProvider
		\item Vor- und Nachbereitung der Tests (\emph{setUp()} und \emph{tearDown()})
		\item Mock-Objekte als Platzhalter für echte Objekte
	\end{itemize}

	\begin{alertblock}{BroadcastReceiver}
		Android stellt keine Bibliothek für das Testen von BroadcastReceivern 
		bereit. Daher muss die Komponente getestet werden, die das Intent an den Receiver sendet. 
		Dabei überprüft man, ob der Receiver korrekt antwortet bzw. reagiert.
	\end{alertblock}
\end{frame}

\begin{frame}
	\frametitle{Applikation testen}
	\begin{itemize}
		\item \emph{ApplicationTestCase} testet Applikation ansich
		\item Starten und Beenden der Applikation
		\item Kein Zugriff auf einzelne Komponenten
		\item Validierung der Angaben in Manifest
	\end{itemize}
\end{frame}

\begin{frame}
	\frametitle{Assertions}
	\begin{itemize}
		\item Android SDK bietet aus JUnit bekannte Assertions
		\item Komplexere Überprüfungen durch \emph{MoreAsserts}
		\item View bezogene Assertions durch \emph{ViewAsserts}\\
			$\rightarrow$ Dient Prüfung von Bemaßung und Positionierung
	\end{itemize}
\end{frame}

\begin{frame}
	\frametitle{Mock-Objekte}
	\begin{itemize}
		\item Minimierung der Abhängigkeiten durch \emph{Dependency Injection}
		\item Verschiedene Komponenten, wie \emph{Context}-, \emph{ContentProvider}- oder  
			\emph{Service}-Objekte
		\item Teilweise sogar Nachbildung durch Mock-Intents 
		\item Leider nur überschriebene Methoden\\
			$\rightarrow$ Werfen \emph{UnsupportedOperationException}\\
			$\rightarrow$ Ableitung der Klassen nötig
		\item Interessant \emph{MockContentResolver} sind vorerst keinem Provider zugeordnet\\
			$\rightarrow$ Explizite Zuweisung mit \emph{ContentResolver.add(String, ContentProvider)}
	\end{itemize}
	
	\begin{table}[t]
		\begin{center}
		   \begin{tabular}{c|c|c|c}
		      MockApplication & MockContext & MockContentProvider \\
		      \hline
		      MockContentResolver & MockPackageManager & MockResources\\
		      \hline
		      MockCursor & MockDialogInterface &  \\
		   \end{tabular}
		   \label{tab:mock_objects}
		   \caption{Mock-Objekte}
		\end{center}
	\end{table} 
\end{frame}

\begin{frame}
	\frametitle{Mock-Context}
	\begin{itemize}
		\item Nachbildung globaler Schnittstellen
		\item Zwei Context-Mock-Objekte
			\begin{description}
				\item[IsolatedContext] Klasse stellt einen isolierten Kontext zur Verfügung, 
					der Operationen auf Dateien, Datenbanken oder Verzeichnissen 
					in Testumgebung ausführt, was den Funktionsumfang einschränkt.
				\item[RenamingDelegatingContext] Stellt einen Kontext zur Verfügung, der 
					fast alle Funktionen durch ein normales \emph{Context}-Objekt ausführen lässt und 
					nur Datei- und Datenbank-Operationen in einem \emph{IsolatedContext} 
					ausführt.
			\end{description}
	\end{itemize}
\end{frame}

\section{Tests ausführen}

\begin{frame}
	\frametitle{Allgemeines}
	\begin{itemize}
		\item Ausführung von Tests durch \emph{Test-Runner-Klasse}
		\item Lädt Tests und zu testendes Projekt
		\item Vorbereiten (\emph{setUp()}), Ausführen (\emph{run()}) und Nachbereiten 
			(\emph{tearDown()}) jedes einzelnen Tests
		\item Standard-Runner-Klasse \emph{InstrumentationTestRunner} 
			(erweitert JUnit-Test-Runner)
		\item Deklaration des Test-Runners im Manifest über \emph{\textless{}instrumentation\textgreater}-Element
		\item Laden einer Bibliothek im Manifest mit \emph{\textless{}uses-library\textgreater}-Element
	\end{itemize}
\end{frame}

\begin{frame}
	\frametitle{Das Manifest}
	\lstinputlisting[
		language=xml,caption=Ein Manifest für Test-Projekte,
		label={lst:testManifest.xml}]{src/xml/testManifest.xml}
\end{frame}

\section{Zusammenfassung}
\begin{frame}
	\frametitle{Fazit}
	\begin{itemize}
		\item Unterscheidung zwischen JUnit und Android-JUnit
		\item Zusätzliche Einbindung von Android-Assertions
		\item Test-Klassen für Komponenten
		\item Verschiedene Mock-Objekte
		\item Spezieller Test-Runner
	\end{itemize}
\end{frame}

\begin{frame}
	\frametitle{Probleme}
	\begin{itemize}
		\item Android-JUnit basiert auf JUnit 3 (nicht auf 4)
		\item Tests werden im Emulator ausgeführt
		\item Applikation wird für jeden Test neu gepackt und gestartet\\
			$\rightarrow$ Sehr lange Test-Zeiten
		\item Manche Dinge nur sehr schwer zu testen (Adapter \& Menüs)
		\item Unzureichende Mock-Objekte	
		\item Dalvik VM unterstützt keine Java Reflections 
			(benötigt von bekannten Mocking Frameworks)
		\item \emph{android.jar} enthält nur unvolständige Class-Dateien\\
			$\rightarrow$ Android-Spezifische Klassen werfen auserhalb der Dalvik VM 
			\emph{RuntimeException}
	\end{itemize}
\end{frame}

\begin{frame}
	\frametitle{Alternativen}
	\begin{itemize}
		\item Bekannte Mocking-Frameworks, wie \emph{Mockito}, \emph{EasyMock} 
			oder \emph{jMock} \\
			$\rightarrow$ Benötigen Java Reflections 
		\item Android-Wrapper für EasyMock (\emph{android-Mock}) \\
			$\rightarrow$ Erzeugt die Mock-Objekte bereits zur Compile-Zeit
		\item \emph{Robolectric} ermöglicht Erzeugen von Android-Komponenten 
			direkt in JVM (ohne \emph{RuntimeException}) \\
			$\rightarrow$ Normale Verwendung von \emph{Mockito} oder \emph{EasyMock}\\
			$\rightarrow$ Erlaubt das schreiben normaler JUnit-Tests
	\end{itemize}
\end{frame}

\begin{frame}
	\frametitle{Robolectric-Beispiel}
	\lstinputlisting[
		language=xml,caption=Ein Manifest für Test-Projekte,
		label={lst:testManifest.xml}]{src/xml/testManifest.xml}
\end{frame}

\part{Google Play}
\frame{\partpage}
\begin{frame}
	\frametitle{Contents}
	\tableofcontents[]
\end{frame}

\begin{frame}
	\frametitle{Allgemeines}
	\begin{itemize}
		\item \emph{Google Play Store} ist der offizielle Android-Markt\\
			$\rightarrow$ Ausnahme \emph{Kindle Fire} HD\\
			$\rightarrow$ Erfordert \emph{Publisher Account}\\
			$\rightarrow$ Benötigt \emph{Google Checkout Merchant Account} für Verkäufe
		\item Vertrieb von Applikationen, Musik, Büchern und Filmen
		\item Alternativ privater Vertrieb (Homepage)
		\item Amazon-Market
	\end{itemize}
\end{frame}

\begin{frame}
	\frametitle{Registrierung}
	\begin{itemize}
		\item \emph{Google Play Store} erfordert \emph{Publisher Account}\\
			$\rightarrow$ Einmalig 25\$\\
			$\rightarrow$ \href{https://play.google.com/apps/publish/}{https://play.google.com/apps/publish/} 
		\item Abfrage von Name, Passwort, Geburtsdatum, usw.
		\item Anlegen eines Entwickler-Profils (Name, Website und Telefonnummer)
		\item \emph{Vereinbarung für den Entwicklervertrieb} (landesspezifisch)
		\item Überweisung der 25\$ mit \emph{Google Checkout}
	\end{itemize}
\end{frame}

\begin{frame}
	\frametitle{Checkliste}
	\begin{itemize}
		\item Prüfen ob der Paketname aussagekräftig ist (kann nicht mehr geändert werden)
		\item Entfernen des \emph{android:debuggable}-Attibuts aus dem Android-Manifest
		\item Entfernen aller Debug- und Log-Ausgaben
		\item Sicherstellen, dass keine Test-Ressourcen (beispielsweise Test-Server) 
			mehr verwendet werden
		\item Verzeichnis-Struktur des Projekts bereinigen -- Jedes Verzeichnis sollte nur die 
			für es vorgesehenen Dateien enthalten
		\item Entfernen unnötiger RAW-Dateien aus \emph{assets/} und \emph{res/raw/}
		\item Überprüfung der gesetzten Rechte in Android-Manifest
		\item Hinterlegen eines Namens und eines Icons in Android-Manifest
		\item Android-SDK-Versionen in Android-Manifest überprüfen
		\item \emph{android:versionCode} und \emph{android:versionName} in Android-Manifest prüfen
	\end{itemize}
\end{frame}

\begin{frame}
	\frametitle{Release-APK}
	\begin{itemize}
		\item Signieren mit privatem Schlüssel\\
			$\rightarrow$ Erzeugung mit \emph{Keytool}\\
			$\rightarrow$ Signieren mit \emph{Jarsigner}
		\item Überarbeitung des Pakets mit \emph{zipalign} (SDK-Werkzeuge)
		\item Hochladen des *.apks über \emph{Publisher Konsole}
		\item Einstellung von Vertriebsländern, Preisen
		\item Erstellung von Dokumentation und Screenshots
		\item Kategorisierung
		\item Veröffentlichen
	\end{itemize}
	
	\begin{alertblock}{Google-Prüfung}
		Seit Kurzem werden alle neu in den Play Store hochgeladenen Pakete vor dem 
		Vertrieb durch ein automatisiertes Google-Tool geprüft um 
		Schadprogramme zu filtern.
	\end{alertblock}
\end{frame}
